\documentclass[a4paper,10pt]{article}
\usepackage[margin=0.5in]{geometry}
\usepackage{amsthm}
\usepackage[latin1]{inputenc}
%\usepackage[utf8]{inputenc}
%\usepackage{mathtools}
\usepackage{amsfonts}
\usepackage{float}
%\usepackage{amssymb}
\usepackage{fancyvrb}
\usepackage{indentfirst}
\usepackage{array}
\usepackage{graphicx}
\usepackage[round]{natbib}
\usepackage{array}
\theoremstyle{plain}
\usepackage{color}
\usepackage[boxed,commentsnumbered]{algorithm2e}
\newtheorem{theo}{Theorem}
\newtheorem{defn}{Definition}


\newcommand{\argmin}{\arg\!\min}
\newcommand{\argmax}{\arg\!\max}

\begin{document}
\author{Ieltzu Irazu, Mikel De Velasco Y Mar�a In�s Fernandez}
\pagenumbering{arabic}
\title{Pr�ctica 7}

\author{\thanks{}}
\date{\today}
\maketitle

\section{Introducci�n:}
\subsection{Exposici�n de la pr�ctica:}
En la pr�ctica presentada se nos pide implementar los algoritmos de optimizaci�n denominados b�squeda local(para una sola soluci�n) y algoritmos gen�ticos(varias soluciones) para el conjunto de datos dantzig42. Para ello hemos desarrollado nuestro c�digo en el lenguaje de programaci�n Java y despu�s hemos creado nuestro ejecutador .jar. Adem�s hemos desarrollado este documento para plasmar las partes m�s importantes de la pr�ctica.

\subsection{Objetivos:}
Los objetivos son claros; conseguir un algoritmo eficiente y efectivo para optimizar una solci�n o un conjuntos de soluciones de un conjunto de datos. El problema planteado tiene nombre propio TSP. El c�digo que hemos desarrollado no es dependiente a un solo conjunto de datos,ya que es posible utilizarlo en m�s de un conjunto de datos(menos la lectura).

Como hemos recibido el archivo dantzig42 para desarrollar la pr�ctica, vamos a analizar el archivo para ver su contenido. En el archivo de datos podemos encontrar un conjunto de valores num�ricos separados por ceros. Estos ceros indican un salto de linea. Si procesamos estos datos, conseguimos la mitad una matriz de una dimensi�n de 49x49. Como tenemos 49 ciudades, el punto en com�n de los pares de ciudades nos indica cual es el peso de una ciudad a otra.

Para optimizar nuestras ciudades vamos a aplicar distintos criterios en el proceso. Empezaremos utilizando el criterio Best First y luego utilizaremos el Greedy, para la b�squeda local. Una vez conseguidos estos procesos, haremos el calculo de el algoritmo gen�tico.  

\section{Pseudoc�digo del algoritmo:}

\begin{algorithm}[H] 
\caption{B�squeda Local} 
\SetKwInOut{Input}{entrada} 
\SetKwInOut{Output}{salida}
\Input{} 
\Output{ListaOrdenadaCiudades} 
\While{$D_{Pos} = D_{Grow - Pos} \cup D_{Prune - Pos} \neq \emptyset$}{
	 \tcp*[r]{Construir una nueva regla}
	 Dividir $D$ en $(D_{Grow - Pos} \cup D_{Grow - Neg}) \cup (D_{Prune - Pos} \cup D_{Prune - Neg})$ \\
	 Rule $_{:=}$ GrowRule($D_{Grow - Pos} \cup D_{Grow - Neg}$)\\
	 Rule $_{:=}$ PruneRule($D_{Prune - Pos} \cup D_{Prune - Neg}$)\\
	 \eIf{la tasa de error de Rule en $(D_{Prune - Pos} \cup D_{Prune - Neg}) > 50\%$}{
	 	\Return RuleSet
	 }{
	 	A�adir Rule a RuleSet\\
		Borrar ejemplos cubiertos por Rule de $D$
		
	 }
}
\Return RuleSet
\end{algorithm} 

\begin{algorithm}[H] 
\caption{Busqueda Local} 
\SetKwInOut{Input}{entrada} 
\SetKwInOut{Output}{salida}
\Input{Secuencia$_1,...,$ secuencia$_n$.} 
\Output{RuleSet} 
\While{$D_{Pos} = D_{Grow - Pos} \cup D_{Prune - Pos} \neq \emptyset$}{
	 \tcp*[r]{Construir una nueva regla}
	 Dividir $D$ en $(D_{Grow - Pos} \cup D_{Grow - Neg}) \cup (D_{Prune - Pos} \cup D_{Prune - Neg})$ \\
	 Rule $_{:=}$ GrowRule($D_{Grow - Pos} \cup D_{Grow - Neg}$)\\
	 Rule $_{:=}$ PruneRule($D_{Prune - Pos} \cup D_{Prune - Neg}$)\\
	 \eIf{la tasa de error de Rule en $(D_{Prune - Pos} \cup D_{Prune - Neg}) > 50\%$}{
	 	\Return RuleSet
	 }{
	 	A�adir Rule a RuleSet\\
		Borrar ejemplos cubiertos por Rule de $D$
		
	 }
}
\Return RuleSet
\end{algorithm} 

\section{Experimentaci�n y Resultados:}
Hemos ejecutado el programa y nos han salido estos resultados:

\begin{figure}[H]
\caption{gr�fico 1} 
\centering
\includegraphics[width=0.6\textwidth]{./grafica1.png}
\end{figure}

Aqu� escribimos conclusiones\\
\begin{figure}[H]
\caption{gr�fico 2}
\label{figura1} 
\centering
\includegraphics[width=0.6\textwidth]{./grafica2.png}
\end{figure}
En cuanto al gr�fico del tiempo vemos que cuanto m�s grande es el par�metro k el tiempo de ejecuci�n para cada partici�n es mayor. Esto quiere decir, que cuantos m�s clusters existan, m�s distancias se deben de calcular entre los centroides y las instancias, por lo tanto, un mayor tiempo de ejecuci�n. 
\section{Conclusiones:}
El algoritmo K-Means tiene una gran limitaci�n, y es que es muy dependiente del conjunto de datos. Es un algoritmo sencillo de implementar pero pese a su simplicidad es bastante eficiente. Los algoritmos de evaluaci�n interna SSE(p) y Silhouette son bastante completos, aunque a nuestro parecer el Silhouette es el mejor de los dos.
	
\section{Valoraci�n Subjetiva:}
\textbf{Ieltzu}:  Ha sido una pr�ctica en la que no hemos invertido mucho tiempo. Los tres ten�amos que haber estudiado los dos algoritmos, por lo tanto, no ha sido muy complicado implementarlo. Pr�ctica interesante, pero en la �poca del a�o que nos ha pillado, para mi, sobraba.\\
\textbf{Mikel}:


\textbf{Maria}:



\section*{Bibliografia}
\begin{itemize}
	\item http://www.herrera.unt.edu.ar/gapia/Curso_AG/Curso_AG_08_Clase_5.pdf
\end{itemize}

\end{document}

